\begin{table}[H]
\centering
\begin{tabular}{|c|c|>{\centering\arraybackslash}m{1.5cm}|>{\centering\arraybackslash}m{2cm}|c|c|>{\centering\arraybackslash}m{2cm}|}
\hline
Måned & Køb/salg & Antal enheder & Lager- beholdning & Pris & Resultat & Akkumuleret resultat \\ \hline
0 & -- & -- & 5 & -- & 0 & 0 \\
1 & Køb & 3 & 8 & 19,93 & -59,79 & -59,79 \\
2 & Salg & -4 & 4 & 21,85 & 87,40 & 27,61 \\
3 & Salg & -4 & 0 & 24,75 & 99,00 & 126,61 \\
4 & Køb & 2 & 2 & 17,76 & -35,52 & 91,09 \\
5 & Køb & 4 & 6 & 14,75 & -59,00 & 32,09 \\
6 & Køb & 4 & 10 & 14,70 & -58,80 & -26,71 \\
7 & Salg & -4 & 6 & 19,54 & 78,16 & 51,45 \\
8 & Køb & 4 & 10 & 18,50 & -74,00 & -22,55 \\
9 & Salg & -4 & 6 & 20,38 & 81,52 & 58,97 \\
10 & Køb & 4 & 10 & 11,16 & -46,44 & 12,53 \\
11 & -- & 0 & 10 & 21,21 & 0 & 12,53 \\
12 & Salg & -4 & 6 & 24,05 & 96,08 & 108,61 \\
Slut & Salg & -6 & 0 & 24,05 & 144,12 & \textbf{252,73} \\ \hline
\end{tabular}
\caption{Tabel over strategi for køb og salg.}
\label{tab:kob_salg_strategi}
\end{table}

\autoref{tab:kob_salg_strategi} viser vores strategi over hvornår vi planlægger at købe og sælge, for at optimere vores resultat. Strategien tager udgangspunkt i vores graf i \autoref{fig:gaslager_graf} og priserne fra det givne materiale. Priserne er sat op efter en diskonterinsfaktor på 0,04 pr. måned sammenlignet med vores nutidsværdi. Da de laveste priser forekommer i måned 5, 6 og 10, sørger algoritmen for, at kunne købe $i_{\max}$ enheder i denne periode, på samme måde er priserne højest i måned 2, 3, og 12/slut, så der sørges igen for at kunne sælge så mange som muligt, i tilfældet af 12/slut er dette, at der er 10 på lager i slutningen af måned 11, da vi kan sælge helt ud af lageret i dette tidspunkt.
