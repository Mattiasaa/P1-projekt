\chapter{Indledning}
Klimakrisen har efterhånden været en del af den offentlige og politiske debat længe. For virksomheder kan det være svært at balancere mellem at være effektiv eog klimavenlig. Dette kan være en svær balance at finde, da klimavenlige alternativer ikke altid er den løsning, der giver mest profit. Derfor er det ikke altid, at det mest klimavenlige alternativ er specielt holdbart for virksomheder, men kan der findes et kompromis? \\ \\Hvis vi kigger på naturgas, indeholder det store mængder af metan, hvilket betyder en stor mængde af brint, og knap så meget CO2 udslip. (https://www.experimentarium.dk/klima/naturgas/) Der udledes 40$\%$ mindre $CO_{2}$, sammenlignet med kul, ved den samme mængde energi. Selvom naturgas er et fossilt brændstof, er det altså mere miljøvenligt end kul og olie. Men for at få virksomheder til at benytte sig af denne form for energi, modsat kul og olie, er det vigtigt at gøre det profitabelt for dem. Dette kan gøre ved at optimere deres processer. \\ Procesoptimering kan gøres på et utal af måder, blandt andet ved at implementere en algoritme, der kan udregne, hvornår det er bedst at købe og sælge gassen, og hvor meget ad gangen der skal sælges. \\
\\I dette projekt vil vi arbejde med sådan en algoritme og bruge den til at optimere et gaslager, som vi har lejet for et år. Vi skal i denne periode forsøge at maksimere profitten ved at købe og sælge gasenheder til forskellige tider med forudbestemte priser. Vi vil både kigge det som et basisproblem med bestemte begrænsninger, og derefter generalisere vi det ved at udvide problemet. Vi har dermed problemstillingen: 

\section{Problemanalyse}
I denne rapport ønsker vi at optimere et gaslager ved at få den størst mulige profit for det år vi lejer den. Med problemet blev der også givet visse begrænsninger for gaslageret herunder en nedre og en øvre grænse for, hvor meget gas lageret kan indeholde, samt en nedre og en øvre grænse for hvor meget gas der må købes og sælges hver måned. Der er dermed tale om et optimeringsproblem. Dette problem kan beskrives matematisk ved hjælp af grafteori. Derfor vil der i dette projekt blive redegjort for centrale aspekter inden for grafteori såsom
graftyper, repræsentation af grafer, veje og delte grafer. Med denne teori vil vi være i stand til at opstille en graf for vores problem. Vi vil dermed kunne tælle os frem til hvilken vej der giver den største profit, men da dette er en langsommelig proces, ønsker vi at opstille en algoritme, der kan udføre dette arbejde. \\ Derfor vil vi også beskrive centrale elementer indenfor algoritmer. Vi vil først komme ind på nogle forskellige algoritmetyper. Herefter vil vi gennemgå Dijkstras algoritme, og til sidst vil vi komme ind på kompleksitet, der kan bruges til at finde køretiden for Dijkstras algoritme.

\section{Problemformulering}
Dette leder os frem til følgende problemformulering:
\textit{Hvordan kan vi optimere profitten for et gaslager ved brug af grafteori og algoritmer, og kan vi generelisere problemet til også at omfatte andre virksomheder ved at udvide problemet?}


