\chapter{Indledning}
\section{Problemanalyse}
Klimakrisen har efterhånden været en del af den offentlige og politiske debat i mange år. En stor del af debatten går på, at de klimavenlige alternativer oftest er dyrere end de mindre klimavenlige alternativer. Dette er problematisk, da det betyder, at vi er nødt til at træffe valg, om hvad vi synes er vigtigst. På lang sigt er dette muligvis klimaet, men når man står og skal træffe valget, ender mange mennesker alligevel med at gå med det billige valg, da dette gavner personen mest i øjeblikket.
For virksomheder kan det især være svært at balancere mellem at være økonomisk effektiv og samtidig klimavenlig. Klimavenlige alternativer er nemlig ikke altid den mest profitable løsning, og tjener virksomhederne ikke nok på deres produkter, kan det ende med at forårsage en nedlukning. Derfor vurderes det mest klimavenlige alternativ ikke altid til at være specielt holdbart for virksomheder, men kan der findes et kompromis? 

Virksomhederne er i høj grad nødt til at finde en balance mellem at tilgodese klimaet og finde økonomisk profitable løsninger på enerimarkedet. Mange virksomheder benytter sig af de billigere og mere forurenende alternativer, men ville dette ændre sig, hvis de mere klimavenlige alternativer var billigere?
Hvis vi kigger på naturgas, indeholder det store mængder af metan, hvilket betyder en stor mængde af brint og knap så meget CO2 udslip. (https://www.experimentarium.dk/klima/naturgas/) Der udledes 40$\%$ mindre $CO_{2}$ sammenlignet med kul ved den samme mængde energi. Selvom naturgas er et fossilt brændstof, er det altså mere miljøvenligt end kul og olie, men for at få virksomheder til at benytte sig af denne form for energ, er det vigtigt at gøre det profitabelt for dem. Dette kan gøres ved at optimere naturgaslagres processer, så de kan få en større profit pr. gasenhed, de sælger. Hvis de tjener mere pr. gasenhed kan de nemlig sænke priserne, og på den måde bliver de mere konkurrencedygtige på energimarkedet over for de mindre miljøvenlige alternativer.
Procesoptimering kan gøres på et utal af måder blandt andet ved at implementere en algoritme, der kan udregne, hvornår det er bedst at købe og sælge gassen, og hvor meget der skal købes eller sælges ad gangen. 

I dette projekt vil vi arbejde med en sådan algoritme og bruge den til at optimere et naturgaslager, som vi vil leje for et år. Vi skal i denne periode forsøge at maksimere resultatet ved at købe og sælge gasenheder til forskellige tider med forudbestemte priser. Vi tager udgangspunkt i basisproblemet, men sørger for at algoritmen er optimeret således, at den er generel nok til at løse eventuelle udvidelser uden store ændringer.

\subsection{Problemformulering}
Dette leder os frem til følgende problemformulering:
\textit{Hvordan kan vi optimere profitten for et gaslager ved brug af grafteori og algoritmer, og kan vi generalisere løsningen ved at udvide problemet?}

\section{Introduktion}
I denne rapport ønsker vi at optimere et gaslager ved at få den størst mulige profit, det år vi vil leje det. Med problemet blev der også givet visse begrænsninger for gaslageret, herunder en nedre og en øvre grænse for, hvor meget gas lageret kan indeholde samt en nedre og en øvre grænse for, hvor meget gas der må købes og sælges hver måned. Der er dermed tale om et optimeringsproblem. 

Problemet kan beskrives matematisk ved hjælp af grafteori. Derfor vil der i dette projekt blive redegjort for centrale aspekter inden for grafteori såsom graftyper, repræsentation af grafer, veje, vægtede grafer og delte grafer. Med denne teori vil vi være i stand til at opstille en graf for vores problem. Vi vil dermed kunne finde den vej, der giver den største profit. 
Dette kan dog være en langsommelig proces, hvis grafen er stor nok. Derfor ønsker vi at opstille en algoritme, der kan udføre dette arbejde for os. Vi vil derfor også beskrive centrale elementer indenfor algoritmer. Vi vil først komme ind på nogle forskellige algoritmetyper. Herefter vil vi gennemgå Dijkstras algoritme, som bruges til at finde den korteste vej i en graf. Til sidst vil vi komme ind på kompleksitet og NP problemer, som giver et indblik i effektiviteten af algoritmer.

Til sidst vil vi benytte teorien om grafer og algoritmer til at løse vores optimeringsproblem. Vi giver først eksempler på, hvordan et problem af denne type løses med grafteori og Dijkstras algoritme. Derefter opstiller vi vores egen algoritme med udgangspunkt i Dijkstras algoritme. Denne algoritme er tilpasset vores problem og kan dermed finde den længste og mest profitable vej igennem vores opstillede graf. Derudover beregner den resultatet, man får, når man følger denne vej. Dette gøres både for basisproblemet og det udvidede problem.




