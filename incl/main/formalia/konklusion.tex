\chapter{Konklusion}
Procesoptimering er en vigtig del af, hvordan virksomheder øger deres overskud. Dette gælder også for ejere af gaslagre, som ved køb og salg af gas skal forsøge at få det største resultat. På baggrund af grafteori og viden om algoritmer, har vi opstillet en graf for et gaslagers mulige køb og salg i løbet af det år, vi vil leje lageret. For at få det største overskud, har vi optimeret vores handelssstrategi. 

For at gøre dette, har vi udarbejdet en algoritme i Python 3, som kan finde den vej gennem den omtalte graf, der giver det største resultat. Denne vej viser ændring i lagerbeholdning over tiden, $t$, hvilket giver os vores strategi. For vores basisproblem er det maksimale resultat, ud fra de givne værdier i opgaven, 252,73€. Denne opnås ved at følge handelsstrategien givet i \autoref{tab:kob_salg_strategi}. Dette er også illustreret på \autoref{fig:gaslager_graf}. 

Vores algoritme er tilpasset, så den også kan anvendes på vores udvidede problem. I vores udvidede problem skal vi tage højde for, at $q_{\min}$, $q_{\max}$, $i_{\max}$ og $u_{\max}$ ikke er statiske, men varierende alt efter hvilken måned vi befinder os i. Derudover er der i det udvidede problem også en straffaktor, hvis lagerbeholdningen til sidst ikke er lig en aftalt slutbeholdning, $q_{\goal}$. Dette betyder, at gaslagerets ejer ikke vil købe gassen tilbage for fuld pris, hvis ikke vi rammer $q_{\goal}$. Det største resultat, vi kan få med det udvidede problem, er 188,27€. Dette opnås ved at følge handelsstrategien givet i \autoref{tab:kob_salg_strategi_udvidet}, som er opstillet ud fra \autoref{fig:gaslager_graf_udvidet}.
Det ses altså, at udvidelserne mindsker resultatet betragteligt, og især de svingende grænser for $q_{\min}$, $q_{\max}$, $i_{\max}$ og $u_{\max}$ har betydning, da de påvirker overskuddet over flere måneder og ikke kun de sidste to måneder, som straffaktoren gør. 
Udvidelserne gør altså investeringen i leje af gaslageret mindre attraktiv.

Da der ikke er taget højde for hverken lønomkostninger eller omkostninger i forbindelse med leje af gaslageret, fx leje af lokaler, el, vand og varme, vil det være svært at vurdere, om det er lønsomt at overtage gaslageret i det givne år. Ud fra parametrene, som er opstillet, vil det være en god investering at leje dette gaslager. 


