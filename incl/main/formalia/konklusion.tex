\chapter{Konklusion}
Procesoptimering er en vigtig del af, hvordan virksomheder øger deres profit. Dette gælder også for ejere af gaslagre, som ved at købe og sælge gas skal forsøge at balancere mellem at være konkurrencedygtige på markedet med fornuftige priser og samtidig tjene mest muligt. På baggrund af grafteori og viden om algoritmer, har vi optimeret overskuddet af gaslageret. På baggrund af teorien bag disse emner har vi opstillet en graf for et gaslagers mulige køb og salg i løbet af det år, vi vil leje lageret. 

Derudover har vi udarbejdet en algoritme i programmet Python 3, som kan optimere handelsstrategien for gaslageret ved at finde den vej gennem den omtalte graf, der giver det største resultat. For vores basisproblem gælder det, at det maksimale resultat ud fra de givne værdier i opgaven er 252,73 euro. Denne opnås ved at følge handelsstrategierne givet i \autoref{tab:kob_salg_strategi}. Dette er også illustreret på \autoref{fig:gaslager_graf}. 

Vores algoritme kan med få ændringer tilpasses, så den også kan anvendes på vores udvidede problem. I vores udvidede problem skal vi tage højde for, at lageret ikke har faste grænseværdier, men minimums- og maksimumsbeholdningen svinger alt efter, hvilken måned vi befinder os i. På samme måde ændrer antallet af gasenheder, som vi må købe og sælge, sig også. Derudover er der i det udvidede problem også en straffaktor, hvis lagerbeholdningen til sidst ikke er $q_{\goal}$. Dette betyder, at gaslagerets ejer ikke vil købe gassen tilbage for fuld pris, hvis ikke vi har ramt $q_{\goal}$. Det største resultat, vi kan få med det udvidede problem, er 188,27 euro. Dette opnås ved at følge handelsstrategien givet i \autoref{tab:kob_salg_strategi_udvidet}. Dette er også illustreret på \autoref{fig:gaslager_graf_udvidet}.


Det ses altså, at udvidelserne mindsker resultatet betragteligt, og især de svingende grænser for $q_{\max}$, $q_{\min}$, $u_{\max}$ og $i_{\max}$ har betydning, da de påvirker overskuddet over flere måneder og ikke kun i den sidste måned, som straffaktoren gør. Hvis man ønsker at maksimere sit resultat, skal man derfor forsøge at mindske nogle af disse udvidelser, hvis dette er muligt.

Da der ikke er taget højde for hverken lønomkostninger eller omkostninger i forbindelse med leje af gaslageret, leje af lokaler, el, vand og varme, med flere, vil det være svært at vurdere, om det er lønsomt at overtage gaslageret i det givne år. Ud fra parametrene, som er opstillet, vil det være en god investering at leje dette gaslager. 


