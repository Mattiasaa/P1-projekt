\chapter{Konklusion}
Procesoptimering er en vigtige del af, hvordan virksomheder øger deres profit. Dette gælder også for ejere af gaslagre, som ved at købe og sælge gas skal forsøge at balancere mellem at være konkurrencedygtige på markedet med fornuftige priser og samtidig tjene mest muligt. Denne balance har vi forsøgt at finde ved at arbejde med grafteori og algoritmer. På baggrund af teorien bag disse emner har vi opstillet en graf for et gaslagers mulige køb og salg i løbet af det år vi lejer lageret. \\Derudover har vi udarbejdet en algoritme i programmet Python, som kan opstimere handelsstrategien for gaslageret ved at finde den vej gennem den omtalte graf, der giver den største profit. For vores basisproblem gælder det, at den maksimale profit ud fra de givne værdier i opgaven er 252,72 euro. Denne opnås ved at følge handelsstrategierne givet i (Figur ?). Dette er også illustreret på (Graf ?) \\
\\
Vores algoritme kan med få ændringer tilpasses så den også kan anvendes på vores udvidede problem. I vores udvidede problem skal vi tage højde for at lageret ikke har faste grænseværdier, men minimums- og maksimumsindholdet svinger alt efter, hvilken måned vi befinder os i. På samme måde ændrer antallet af gasenheder, som vi må købe og sælge, sig. Derudover er der i det udvidede problem også en straffaktor, hvis lagerbeholdningen til sidst ikke er $q_goal$. Dette betyder at gaslagerets ejer ikke vil købe gassen tilbage for fuld pris, hvis ikke vi har ramt $q_goal$. Den største profit, som vi kunne få med det udvidede problem var 188,27 euro. Denne opnås ved at følge handelsstrategien givet i (Figur ?). Dette er også illustreret på (Graf ?)
\\
\\
Det ses altså, at udvidelserne mindsker profitten betragteligt, og især de svingende grænser for $q_max$, $q_min$, $u_max$ og $i_max$ har betydning, da de påvirker profitten over flere måneder og ikke kun i den sidste måned, som straffaktoren gør. Hvis man ønsker at maksimere sin profit, skal man derfor forsøge at mindske nogle af disse udvidelser, hvis dette er muligt.


