\section{Det udvidede problem} \label{kap:udvidet_problem}

I basisproblemet arbejder vi med mange statiske værdier. Vi siger, at $q_{\min}$, $q_{\max}$, $i_{\max}$ og $u_{\max}$ alle er statiske igennem hele lejeperioden, men en mere realistisk antagelse er, at disse er variable. Eksempelvis vil $q_{\min}$ typisk være højere end nul om vinteren for at sikre, at man i situationer, hvor der pludselig ikke er mulighed for at anskaffe mere gas, stadig kan imødekomme den forøgede efterspørgsel på strøm. Derfor udvider vi nu basisproblemet ved at tilføje tidsafhængige begrænsninger, altså\\
1. $q_{\min,t} \leq q_{t} \leq q_{\max,t}$ \\
2. $\Delta q_{t} \in \{-u_{\max,t}, -u_{\max,t} + 1, \dotsc, -1, 0, 1, \dotsc, i_{\max,t} -1, i_{\max,t} \}$ \\
for $t \in \{1,2,\dotsc,T \}$. Dermed afhænger disse værdier nu af $t$ modsat basisproblemet, hvor de var statiske hele året. 

Den anden udvidelse til basisproblemet er, at der i lejekontrakten indgår en straffaktor. Det vil sige, at vi efter de 12 måneder skal aflevere lageret med en aftalt mængde gas, $q_{\goal}$. Hvis vi ikke formår at opfylde dette krav, reduceres den pris, ejeren af gaslageret køber den resterende mængde gas for med denne straffaktor, og vi får dermed ikke den fulde pris for gassen. Med denne udvidelse er der altså to nye parametre:
\begin{itemize}
\item $\kappa$, en straffaktor for prisen, hvor $0 < \kappa < 1$
\item $q_{\goal}$, den aftalte mængde gas lagret til tid T
\end{itemize}
Dermed er maksimeringsproblemet med udvidelsen nu:
\begin{equation}
\max_{\Delta p_{t}} \Bigg\{ -\sum_{t=1}^{T} \mathrm{e}^{-r\frac{t}{T}} \Delta q_{t} p_{t}+ \mathrm{e}^{-r}q_{T}p_{T} \left (1- \kappa \cdot \ind [q_{T} \neq q_{\goal}] \right )  \Bigg\}, 
\end{equation}
hvor $\ind [\cdot]$ er en indikatorfunktion og $t \in \{1,2,\dotsc,T \}$. Den er altså lig med 1, når $q_{T} \neq q_{\goal}$, og 0 når de er lig hinanden. Al data til det udvidede problem kan ses i \autoref{code:data_udvidet}.