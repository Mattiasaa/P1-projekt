\subsection{Vægtede grafer}
En vægtet graf er en graf hvori kanterne eller knuderne får tildelt en numerisk værdi. I dette projekt arbejdes der udelukkende med vægtede kanter, og vi vil derfor kun fokusere på dem i dette afsnit.
En vægtet graf er defineret ved:
\begin{defn}[Vægtede grafer]
En vægtet graf $G=(V,E,\psi,\omega)$,består af mængden af knuder $V$, mængden af kanter $E$, incidensfunktionen $\psi: E \rightarrow \{\{u;v\}|u,v \in V\}$ og vægtfunktionen $\omega: E \rightarrow \R$
\end{defn}
For en vægtet graf har alle kanter $e\in E$ en numerisk vægt tildelt, givet ved funktionen $\omega (e)$. Da $e$ er en kant incident med ${u,v}$ kan man  ligeledes skrive $\omega (u,v)$.
\\ Da vægtede grafer har en tildelt numerisk vægt på hver kant, kan man således beregne længden fra en knude til en anden i grafen. Længden fra en knude til en anden kan defineres således:
\begin{defn}
Lad $m\in \N$ og $G=(V,E,\psi,\omega)$ være en simpel graf og lad $e_{i,i+1}$ være en kant incident med $i$ og $i+1$. Lad en tilfældig vej $P$ gå gennem knuderne således $P=(v_0,v_1...v_m)$, da kan længden beskrives således:
	\begin{equation*}
	dist(P)=\sum_{i=1}^{m}\omega(e_{i,i+1})
	\end{equation*}  
\end{defn}
Man kan således bruge følgende definitioner af korteste og længste vej i en vægtet graf:


\begin{defn} \label{defn:min.vej} [Korteste vej i vægtet graf]
Lad $G=(V,E,\psi,\omega)$ være en simpel og vægtet graf. Da er længden af den korteste vej, fra en knude $v_0$ til en anden knude $v_m$, defineret som $\alpha(v_0,v_m)$.
\end{defn}

På samme vis defineres længste vej:

\begin{defn} [Længste vej i vægtet graf]
Lad $G=(V,E,\psi,\omega)$ være en simpel og vægtet graf. Da er længden af den længste vej, fra en knude $v_0$ til en anden knude $v_m$, defineret som $\beta(v_0,v_m)$.
\end{defn}
