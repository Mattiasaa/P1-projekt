\subsection{Nabolister}
En måde at repræsentere en graf på er ved at lave en naboliste. Nabolister er tabeller, der giver en oversigt over hvilke knuder, der er forbundet med andre knuder. Dog vil man ikke kunne se, hvis der er parallelle kanter. En naboliste er bygget op således, at knuden, man vil beskrive, er i venstre side af tabellen, og naboknuderne er skrevet i højre side. \\

\begin{figure}[h]
  \centering
  \begin{tikzpicture}
    \node[point] at (1,2) (A) [label=above:\(A\)] {};
    \node[point] at (3,2) (B) [label=above:\(B\)] {};
    \node[point] at (4,1) (C) [label=right:\(C\)] {};
    \node[point] at (2,0) (D) [label=below:\(D\)] {};
    \node[point] at (3,0) (E) [label=below:\(E\)] {};
    \node[point] at (1,1) (F) [label=left:\(F\)] {};
    \node[point] at (0,2) (G) [label=below:\(G\)] {};

    \footnotesize
    \draw (A) -- (G);
    \path (F) edge [bend left] (B);
    \draw (A) -- (F);
    \path (F) edge [bend right] (B);
    \draw (B) -- (D);
    \draw (B) -- (E);
    \draw (B) -- (C);
    \draw (C) -- (E);
    \draw (C) to [out=315,in=45,looseness=50] (C);
    \draw (E) -- (D);
    \draw (D) -- (F);
    \draw (F) -- (G);
  \end{tikzpicture}
  \caption{Ikke-orienteret pseudograf.}
  \label{fig:ikke-orienteret-pseudo}
\end{figure}

\begin{center}
	\begin{tabular}{ |p{4cm}||p{3cm}|}
	 	\hline
 		\multicolumn{2}{|c|}{Naboliste til figur \ref{fig:ikke-orienteret-pseudo}} \\
 		\hline
 		Knuder & Naboknuder\\
 		\hline
 		A & F,G \\
		B & C,D,E,F \\
		C & B,E,C \\
		D & B,E,F \\
		E & B,C,D \\
		F & A,B,D,G \\
		G & A,F \\
 	\hline
 	\label{tab:naboliste} 	
	\end{tabular}
	%\caption{Naboliste til figur \ref{fig:ikke-orienteret-pseudo}
\end{center}
Ud fra tabellen ses, at knuden B har naboknuderne C, D, E og F, men man kan ikke se, at der er en ekstra kant mellem B og F. Dog kan man se, at C har en løkke, da den er nabo til sig selv.

\begin{figure}[H]
  \centering
  \begin{tikzpicture}
    \node[point] at (1,2) (A) [label=above:\(A\)] {};
    \node[point] at (3,2) (B) [label=above:\(B\)] {};
    \node[point] at (4,1) (C) [label=right:\(C\)] {};
    \node[point] at (2,0) (D) [label=below:\(D\)] {};
    \node[point] at (3,0) (E) [label=below:\(E\)] {};
    \node[point] at (1,1) (F) [label=left:\(F\)] {};
    \node[point] at (0,2) (G) [label=below:\(G\)] {};

    \footnotesize
    \path [->] (A) edge [bend left] (G);
    \path [->] (A) edge [bend right] (G); 
    \path [->] (F) edge [bend left] (B);
    \draw [<-](A) -- (F);
    \path [<-](F) edge [bend right] (B);
    \draw [->](B) -- (D);
    \draw [<-](B) -- (E);
    \draw [<-](B) -- (C);
    \draw [->](C) -- (E);
    \draw [->](C) to [out=315,in=45,looseness=50] (C);
    \draw [<-](E) -- (D);
    \draw [->](D) -- (F);
    \draw [->](F) -- (G);
  \end{tikzpicture}
  \caption{Orienteret pseudograf.}
  \label{fig:orienteret-pseudo}
\end{figure}

%\begin{center}
%	\begin{tabular}{ |p{4cm}||p{3cm}|}
%	 	\hline
% 		\multicolumn{2}{|c|}{Naboliste til figur \ref{fig:orienteret-pseudo} \\
% 		\hline
% 		Knuder & Naboknuder\\
% 		\hline
% 		A & G \\
%		B & D,F \\
%		C & B,E,C \\
%		D & E,F \\
%		E & B \\
%		F & A,B,G \\
%		G &  \\
% 	\hline
% 	\label{tab:naboliste1}
%	\end{tabular}
%\end{center}

\ref{tab:naboliste1} viser en oversigt over den orienterede grafs (Figur \ref{fig:orienteret-pseudo}) naboknuder. Kigger man på B og F i tabellen, kan man se, at der er parallelle kanter, da kanterne er orienteret i hver deres retning. Kigger man på A og G, kan man ikke se, at der er parallelle kanter mellem A og G, da begge kanter er orienteret fra A til G.
