\subsection{Nabo-lister}
En måde at repræsentere en graf på er ved at lave en Nabo-liste.Nabo-lister er tabeller der giver en oversigt over hvilke knuder der er forbundet med andre knuder. En Nabo-liste er bygget op således at knuden man vil beskrive er i venstre side af tabellen og nabo knuderne er listet i højre side af tabellen. 

\begin{exmp}
	\begin{figure}
	\centering
	\includegraphics[scale=0.5]{fig/img/Nabolisteeks1}
	\caption{En simpel graf}
	\end{figure}
\begin{center}
	\begin{tabular}{ |p{4cm}||p{3cm}|}
	 	\hline
 		\multicolumn{2}{|c|}{Nabo-liste til figur 4.3} \\
 		\hline
 		Knuder & Nabo-knuder\\
 		\hline
 		A & F,G \\
		B & C,D,E,F \\
		C & B,E \\
		D & B,E,F \\
		E & B,C,D \\
		F & A,B,D,G \\
		G & A,F \\
 	\hline
	\end{tabular}
\end{center}
\end{exmp}