\subsection{Nabo-lister}
En måde at repræsentere en graf på er ved at lave en Nabo-liste.Nabo-lister er tabeller der giver en oversigt over hvilke knuder der er forbundet med andre knuder, dog vil man ikke kunne se hvis der er paralelle kanter. En Nabo-liste er bygget op således at knuden man vil beskrive er i venstre side af tabellen og nabo knuderne er listet i højre side af tabellen. \\

\begin{figure}[h]
  \centering
  \begin{tikzpicture}
    \node[point] at (1,2) (A) [label=above:\(A\)] {};
    \node[point] at (3,2) (B) [label=above:\(B\)] {};
    \node[point] at (4,1) (C) [label=right:\(C\)] {};
    \node[point] at (2,0) (D) [label=below:\(D\)] {};
    \node[point] at (3,0) (E) [label=below:\(E\)] {};
    \node[point] at (1,1) (F) [label=left:\(F\)] {};
    \node[point] at (0,2) (G) [label=below:\(G\)] {};

    \footnotesize
    \draw (A) -- (G) node[edge] {};
    \path (F) edge [bend left] (B);
    \draw (A) -- (F) node[edge] {};
    \draw (B) -- (F) node[edge] {};
    \draw (B) -- (D) node[edge] {};
    \draw (B) -- (E) node[edge] {};
    \draw (B) -- (C) node[edge] {};
    \draw (C) -- (E) node[edge] {};
    \draw (E) -- (D) node[edge] {};
    \draw (D) -- (F) node[edge] {};
    \draw (F) -- (G) node[edge] {};
  \end{tikzpicture}
  \caption{Eksempel på graf fra \citep{dmat}.}
  \label{fig:graf}
\end{figure}

\begin{center}
	\begin{tabular}{ |p{4cm}||p{3cm}|}
	 	\hline
 		\multicolumn{2}{|c|}{Nabo-liste til figur 4.5} \\
 		\hline
 		Knuder & Nabo-knuder\\
 		\hline
 		A & F,G \\
		B & C,D,E,F \\
		C & B,E \\
		D & B,E,F \\
		E & B,C,D \\
		F & A,B,D,G \\
		G & A,F \\
 	\hline
 	\label{tab:Nabo-liste}
	\end{tabular}
\end{center}
Tabellen ovenfor viser nabo-listen over en ikke orienteret simpel graf \ref{fig:4.5}. Venstre side af tabellen viser knuderne fra A til G og højre side af tabellen viser hvilke nabo-knuder de har.Foreksempelvis kan man se at knuden B har nabo-knuderne C,D,E,F. 
