\subsection{Nabo-matricer}
En anden mulighed for at repræsentere en graf er ved brug af Nabo-matricer. Nabo-matricer er bedre at bruge når grafen har mange kanter.\\
En nabo-matrix kan beskrives som en $N=n X n$ matrix. hvor $n$ er afhænger af knudemængden $V=\{v_0, v_1, \ldots, v_n\}$. Hvis man har vilkårlig graf $G=(V,E)$ vil der opstå en $m X m$ nul-$k$ matrix hvor $k$ vil være mængden af kanter fra et vilkårlige punkt $v_i$, som har en kant med et andet vilkåligt punkt,$v_j$ , hvis de derimod ikke har en kant vil den få notation 0. \\

Det kan også skrives som \\

\begin{equation}
\begin{Bmatrix} 
	 n \hspace{0.3cm}\textrm{hvis} \hspace{0.3cm}\{v_i,v_j\} \hspace{0.2cm} \textrm{er en kant i grafen} \hspace{0.2cm} G \\
	 \textrm{0 ellers} \\
	\end{Bmatrix}
\end{equation}

hvis man skulle lave en matrice til grafen \ref{fig:nabograf} 

	

