\subsection{Nabomatricer}
En anden mulighed for at repræsentere en graf er ved brug af nabomatricer. Nabomatricer er bedre, når grafen har mange kanter.\\
En nabomatrice kan beskrives som en $N=m X m$ matrice, hvor $m$ er afhængig af knudemængden $V=\{v_0, v_1, \ldots, v_m\}$. Hvis man har en vilkårlig graf, $G=(V,E)$, vil der opstå en $m X m$ nul-$n$ matrice. Hvis der en vilkårlig knude, $v_i$, som har en kant med et andet vilkålig knude,$v_j$, vil den få notationen 1, hvis de derimod ikke har en kant vil den få notation 0. \\

Det kan også skrives som \\

\begin{equation}
\begin{Bmatrix} 
	 \textrm{1 hvis} \hspace{0.3cm}\{v_i,v_j\} \hspace{0.2cm} \textrm{er en kant i grafen} \hspace{0.2cm} G \\
	 \textrm{0 ellers} \\
	\end{Bmatrix}
\end{equation}

hvis man skulle lave en matrice til grafen  

	

