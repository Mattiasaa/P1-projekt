\section{Veje}
Vi har indtil videre snakket om punkter, og hvordan de som punktsæt forbindes med kanter. I dette afsnit vil vi udvide det til at snakke om veje, som er sekvenser af disse kanter. Hvis der er tale om ikke-orienterede grafer, er veje defineret ved:
\begin{defn}
[Veje] 
Lad $n$ være et ikke-negativt heltal og $G$ en ikke orienteret graf. En vej af længde $n$ fra $u$ til $v$ i $G$ er en sekvens af n kanter $e_{1},e_{2},\cdots,e_{n}$ for $G$, for hvilket der eksisterer en sekvens $x_{0}=u,x_{1},x_{2},\cdots,x_{n-1}$,$x_{n}=v$ af punkter sådan at $e_{i}$ har, for $i=1,2,\cdots,n$, endepunkterne $x_{i-1}$ og $x_{i}$ Når grafen er simpel, betegner vi denne  vej ved dets punktsekvens $x_{o},x_{1},\cdots,x_{n}$ Vejen siges at passere igennem punkterne $x_{o},x_{1},\cdots,x_{n-1}$ eller krydse kanterne $e_{1},e_{2},\cdots,e_{n}$. En vej er simpel, hvis den ikke krydser den samme kant mere end én gang.
\end{defn}
Kigger vi derimod på veje med orienterede grafer, som er det vi beskæftiger os med i problemet, ser definitionen en smule anderledes ud:
\begin{defn}
[Veje] 
Lad $n$ være et ikke-negativt heltal og $G$ en ikke orienteret graf. En vej af længde $n$ fra $u$ til $v$ i $G$ er en sekvens af kanter $e_{1},e_{2},\cdots,e_{n}$ for $G$, sådan at $e_{1}$ er forbundet med $(x_{0},x_{1})$, $e_{2}$ er forbundet med $(x_{1},x_{2})$ og så videre frem til $e_{n}$, som er forbundet med $(x_{n-1},x_{n})$. Her er $x_{0}=u$ og $x_{n}=v$. Hvis alle punktsæt er forbundet med højst én kant per sæt, betegner vi denne  vej ved dets punktsekvens $x_{o},x_{1},\cdots,x_{n}$. En vej er simpel, hvis den ikke krydser den samme kant mere end én gang.
\end{defn}

Antallet af veje mellem to punkter i grafen kan findes ved hjælp af nabomatricer, som vi diskuterede i forrige afsnit.
\begin{thm}
[Antallet af veje mellem to punkter] 
Lad G være en graf med nabomatricen
\textbf{$A$} med grafens punkter i rækkefølgen $v_{1},v_{2},\cdots,v_{n}$ (både orienterede og ikke-orienterede kanter samt flere kanter pr punktpar og løkker er tilladt). Antallet af forskellige veje med længde $r$ fra $v_{i}$ til $v_{j}$ vil da være lig den $(i,j)$'te indgang af \textbf{$A^{r}$}.
\end{thm}

\begin{proof}
Bevis: Lad G være en graf med nabomatricen 
\textbf{$A$}, hvor vi antager, at punkterne i $G$ har rækkefølgen $v_{1},v_{2},\cdots,v_{n}$. Antallet af veje fra $v_{i}$ til $v_{j}$ af længde 1 er da den $(i,j)$'te indagang til 
\textbf{$A$}. Dette skyldes, at det blot er antallet af kanter fra $v_{i}$ til $v_{j}$.
Vi antager at den $(i,j)$'te indagang til 
\textbf{${A^r}$} er antallet af forskellige veje som går fra $v_{i}$ til $v_{j}$ og som har længden $r$. Dette er hypotesen, vi ønsker at bekræfte.
Vi ser på nabomatricen \textbf{$A^{r+1}$}. 
\textbf{$A^{r+1}$} er det samme som 
\textbf{$A^{r}$}$\cdot$\textbf{$A$}, og derfor er den $(i,j)$'te indgang af \textbf{$A^{r+1}$} lig med $b_{i1}a_{1j} + b_{i2}a_{2j} +\cdots+ b_{in}a_{nj}$. Her er $b_{ik}$  den $(i,k)$'te indgang til 
\textbf{$A^{r}$}, som ifølge vores hypotese er antallet af veje fra $v_{i}$ til $v_{k}$ med længde $r$.
En vej af længde $r + 1$ fra $v_{i}$ til $v_{k}$ er lavet ud fra en vej med længden $r$ fra begyndelsespunktet $v_{i}$ og hen til et mellemliggende punkt $v_{k}$ samt den kant, der går fra $v_{k}$ til $v_{j}$. Vi ved fra kombinatorik at antallet af muligheder er lig prduktet af mulighederne ved første udfald og mulighederne ved andet udfald. Vi betegner antallet af veje med længden $r$ fra $v_{i}$ til $v_{k}$ med $b_{ik}$ og antallet af kanter fra $v_{k}$ til $v_{j}$ med $a_{kj}$ Finder vi produktet af dette for alle mellemliggende punkter, $v_{k}$, fås det ønskede resultat.
\end{proof}
Vi kan nu opstille et eksempel. 
(((indsæt graf og matrix)))
Vi ønsker, at finde ud af hvor mange veje, der går fra $a$ til $e$ med en længde på 4. Det ses i nabomatricen, at $a$ har 3 naboer, nemlig $b$, $c$ og $d$. Fortsætter vi, kan vi se, at $b$ har $a$ og $d$ som naboer, $c$ har $a$ og $d$, og $d$ har $a$, $b$ og $e$ som naboer. Fortsætter vi, så vi finder alle tænkelige veje med længder på 3, får vi, at der er 18 forskellige veje, der alle starter i a og har en længde på 4. Vi skal nu finde de veje der ved at tilføje en kant ender i $e$. Vi kan se, at $e$ har $c$ og $d$ som naboer. Vi finder derfor de veje der starter i $a$ og slutter i $c$ med længden 3 og derefter dem der slutter i $d$ med længden 3. På denne måde udnytter vi, hvad vi skrev i beviset, nemlig at
\textbf{$A^{r+1}$} er lig med $b_{i1}a_{1j} + b_{i2}a_{2j} +\cdots+ b_{in}a_{nj}$
Her er $b_{ik}$ antallet af veje fra $v_{i}$ til ${v_k}$. I vores eksempel er $v_{i}=a$, ${v_k}=c$ og \textbf{$A^{r+1}$}=\textbf{$A^{3+1}$}. Antallet af veje fra $a$ til $c$ er 5 og antallet af veje fra $a$ til $d$ er 6. Vi kan derfor opstille

\textbf{$A^{4}$}$=b_{1i}\cdota_{1j}+b_{2i}\cdota_{2j}=5\cdot1+6\cdot1=11$
Her er b_{1i} antallet af veje fra $v_{i}=a$ til vores første $v_{k}=c$ og $a_{1j}$ er antallet af kanter fra vores første $v_{k}=c$ til vores $v_{j}=e$. På samme måde optræder $b_{2i}$ og $a_{2j}$ for vores andet $v_{k}=d$. Der er altså 11 veje med længden 4 fra punktet $a$ til punktet $e$.

\input{incl/main/grafer/vægtede}