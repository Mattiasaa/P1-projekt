\section{Delte grafer}
I visse tilfælde kan en graf deles op for at optimere en algoritme til løsningen af en given problemstilling.
To mulige grafopdelinger er delgrafer og $k$-delte grafer.

\begin{defn}[Delgraf] \label{defn:delgraf} %subgraph
En \emph{delgraf} af grafen, $G= (V,E)$, er en graf, $D = (W,F)$, skabt af delmængderne af kanterne, $F \subseteq E$, og knuderne, $W \subseteq V$, hvori det gælder $F \subseteq (\{u,v\} | u,v \in W)$.
\end{defn}

En delgraf er \emph{induceret}, hvis mængden af kanter, $F$, indeholder alle kanter fra $E$, hvis knuder indgår i $W$.
Delgrafen kaldes \emph{udspændende} hvis $W=V$. 

\begin{defn}[\emph{k}-delt] \label{defn:k-delt} %k-partite
En graf, $G = (V, E)$, kaldes en \emph{$k$-delt graf}, hvis $V$ kan deles op i $k$ ikke-tomme delmængder, $V_1, V_2,\dotsc, V_k$, således at $V= V_1 \bigcup V_2 \bigcup \dotsc \bigcup V_k$. Foruden gælder det at $V_i \bigcap V_j  = \emptyset \forall i,j$, og $i\neq j$. Samtidigt kan to vilkårlige knuder, $v$ og $u$, kun være naboer, hvis de befinder sig i forskellige delmængder. 
\end{defn}

Som vi nævnte i det forhenværende kapitel, ved vi, at når vi løser korteste vej problemer, finder vi den optimale substruktur i grafen. Dette kan også beskrive så, at den optimale substruktur $D_{ij}$ er fundet hvis den optimale delgraf $D$ er fundet.



