\section{Delte grafer}
I visse tilfælde kan en graf deles op, for at optimere en algoritme til løsningen af problemet i en given problemstilling. \\
To måder at dele en graf op er $k$-delte grafer og delgrafer.

\begin{defn} \label{defn:k-delt} %k-partite
En graf G kaldes en $k$-delt graf $G = (V_1, V_2, \ldots, V_k; E)$, hvis følgende betingelser opfyldes: $V= V_i \cup V_j, V_i \cap V_j = Ø \forall i,j$, og $i\neq j$ 
\end{defn}

En $k$-delt graf består af $k$ disjunkte, ikke-tomme delmængder, $V_1, V_2, \ldots, V_k$, hvor for 2 vilkårlige knuder, $u$ og $v$, kun er forbundet, hvis de er i forskellige delmængder.
\begin{defn}	 \label{defn:delgraf} %subgraph
En delgraf af grafen $G= (V,E)$ er en graf $D = (W,F)$ skabt af delelementerne af kanterne og knuderne: $W \subseteq V og F \subseteq E$.
\end{defn}

Fordelen ved at dele en graf op i forskellige delgrafer er, at når man i $korteste-vej$, eller i bestemte tilfælde $længste-vej$, kan finde den optimale delstruktur hvis den optimale delgraf er fundet. Altså finder man den optimale vej igennem knuderne $(a, c, g, n)$, er den optimale vej også fundet til de mellemliggende knuder også fundet, altså fra $a$ til fx $g$.
