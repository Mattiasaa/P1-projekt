\section{Delte grafer}
I visse tilfælde kan en graf deles op for at optimere en algoritme til løsningen af en given problemstilling.
To mulige grafopdelinger er delgrafer og $k$-delte grafer.

\begin{defn}[Delgraf] \label{defn:delgraf} %subgraph
En delgraf af grafen, $G= (V,E)$, er en graf, $D = (W,F)$, skabt af delmængderne af kanterne, $F \subseteq E$, og knuderne, $W \subseteq V$, hvori $f_n$ er incident med $w_{n-1}$ og $w_n \forall f_n$.
\end{defn}

En delgraf er induceret, hvis mængden af kanter, $F$, indeholder alle kanter fra $E$, hvis knuder indgår i $W$.
Delgrafen kaldes udspændende hvis $W=V$. 

\begin{defn}[K-delt] \label{defn:k-delt} %k-partite
En graf, $G$, kaldes en $k$-delt graf, $G = (V, E)$, hvis følgende betingelser opfyldes: $V= V_i \bigcup V_j \bigcup \dotsc \bigcup V_k, V_i \bigcap V_j  = \emptyset \forall i,j$ og $i\neq j$ .
\end{defn}

En $k$-delt graf består af $k$ disjunkte, ikke-tomme delmængder, $V_1, V_2, \ldots, V_k$, hvor det for to vilkårlige knuder, $u$ og $v$, gælder, at de kun er naboer, hvis de er i forskellige delmængder.


En af fordelene ved at dele en graf op i forskellige delgrafer er, at man i korteste-vej, eller i bestemte tilfælde længste-vej, kan finde den optimale delstruktur, $D_{ij}$, hvis den optimale delgraf, $D$, er fundet. Hvis den optimale vej, $D$, går fra $v_1$ til $v_5$ igennem knuderne $(v_1, v_3, v_4, v_5)$, er den optimale vej til de mellemliggende knuder også fundet. Eksempelvis går den optimale vej, $D_{ij}$, fra $v_1$ til $v_4$ igennem $v_3$.


