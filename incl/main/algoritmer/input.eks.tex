\begin{exmp}
Vi ser igen på en liste, $(3,4,2,5,1)$, som vi gerne vil sortere i rækkefølge, således at de står med stigende værdi. Til at gøre dette, vil vi bruge indskudssorteringsalgoritmen. Illustrationen, som ses nedenfor, viser, at alle understregede tal står i korrekt rækkefølge. Tallet markeret med fed er det næste tal, som algoritmen skal placere i den korrekte liste.

\begin{figure}[H]
\label{fig:indskud}
	\begin{flushleft}
	$i=1: \ (\underline{3},\textbf{4},2,5,1) \rightarrow (\textbf{4}, \underline{3},2,5,1)\rightarrow (\underline{3,\textbf{4}},2,5,1)$ \\
	$i=2: \ (\underline{3,4},\textbf{2},5,1) \rightarrow (\underline{\textbf{2},3,4},5,1) $\\
	$i=3: \ (\underline{2,3,4},\textbf{5},1) \rightarrow (\textbf{5},\underline{2,3,4},1) \rightarrow (\underline{2}, \textbf{5},\underline{3,4},1) \rightarrow (\underline{2,3}, \textbf{5}, \underline{4},1) \rightarrow (\underline{2,3,4,\textbf{5}},1) $ \\
	$i=4: \ (\underline{2,3,4,5},\textbf{1}) \rightarrow (\underline{\textbf{1},2,3,4,5}) $\\
 	\end{flushleft}
\end{figure}

Vi kan se, at algoritmen starter med at sammenligne fire med tre. Den tjekker først om fire er mindre end tre. Da det er større og der ikke er flere tal i listen, placeres fire i slutningen af den korrekte liste. Den tjekker nu det tredje tal i listen og ser med det samme, at det er mindre end det første tal i den korrekte liste. Derfor placeres to på den første plads i den korrekte liste. Den sammenligner nu fem med alle tal i den korrekte liste og placerer fem efter det største tal, den er større end. Dette gør den med alle tallene i listen, og den har til sidst placeret alle tallene i listen i korrekt rækkefølge.


\end{exmp}