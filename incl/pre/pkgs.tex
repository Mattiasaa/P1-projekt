% incl/pre/pkgs.tex : pakker, som bruges i rapporten
% ------------------------------------------------------------------------------
% Makroer distribueres i pakker, som kan indlæses med \usepackage


% Basis-pakker -----------------------------------------------------------------
\usepackage[T1]{fontenc}           % for at kunne vise visse tegn og bogstaver
\usepackage[utf8]{inputenc}        % for at kunne læse visse tegn og bogstaver
\usepackage{babel}                 % sprogindstillinger
\usepackage[square]{natbib}        % litteraturhenvisninger
\usepackage{hyperref}              % diverse funktioner for links og referencer
\usepackage{xpatch}                % til patching af makroer

% Matematik-pakker -------------------------------------------------------------
\usepackage{mathtools}             % matematik-makroer og blokke
\usepackage{amssymb}               % yderligere matematiske symbolder
\usepackage{amsthm}                % blokke til sætninger osv.
\usepackage{bbm}                   % udvidet 'blackboard bold' skriftstil
\usepackage{bm}                    % bedre fed skrift for matematiske symboler

% Figur-pakker -----------------------------------------------------------------
\usepackage[dvipsnames]{xcolor}    % til farvet skrift
\usepackage{graphicx}              % billedhåndtering
\usepackage{booktabs}              % bedre tabeller
\usepackage{caption}               % indstillinger for billedtekst
\usepackage{subcaption}            % gruppering af flere billeder sammen
\usepackage{tikz}                  % generer vektorgrafik med LaTeX-kode
\usepackage[chapter]{algorithm}    % algoritme-figurer
\usepackage[noend]{algpseudocode}  % pseudokode
\usepackage{listings}              % visning af kildekode
\usepackage{pgfplots}			   % mere avancerede grafer
\usepackage{array}				   % til tables
\usepackage{blkarray}		   	   % til matricer med labels
% Styling-pakker ---------------------------------------------------------------
\usepackage{microtype}             % forbedret justering af tekst
\usepackage{emptypage}             % fjern sidetal på tomme sider
\usepackage{fancyhdr}              % indstillinger for sidehoved/-fod
\usepackage{titlesec}              % indstillinger for overskrifter
\usepackage[margin=3cm]{geometry}  % indstillinger for sidemargen
\usepackage[toc,page]{appendix}    % bedre appendicer
\usepackage[some]{background}      % til at indsætte baggrundsbilleder

% Skrifttyper ------------------------------------------------------------------
% http://www.tug.org/pracjourn/2006-1/hartke/hartke.pdf
% http://www.tug.dk/FontCatalogue/
\usepackage{fourier}               % Adobe Utopia som primær skrifttype
\usepackage{DejaVuSansMono}        % DejaVu Sans Mono til bl.a. kildekode

% Inkluderede tredjepartspakker ------------------------------------------------
% Pakken aautitlepage indholder modificerede makroer fra en tredjepart.
% Se incl/pre/ext/aautitlepage.sty for detaljer
\usepackage{incl/pre/ext/aautitlepage}

% Pakke til at lave frames med baggrundsfarve ------------------------------------------------
% Pakke til at lave frames med baggrundsfarve
\usepackage{mdframed}

%Pakke til fancy P
\usepackage[mathscr]{euscript}
\let\euscr\mathscr \let\mathscr\relax
\usepackage[scr]{rsfso}