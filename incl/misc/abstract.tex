% incl/misc/abstract.tex : projektets abstract

This study investigates how algorithms and graph theory can help with the optimization of profit from a gas storage, over the course of a given period of time, by buying and selling gas units. 

In order to give clarification of the calculations and methods used in this study, the essential and relevant graph theory, and theory of algorithms will therefore be explained.
During this study it was found that Dijkstra's algorithm was very useful for solving shortest path problems in graphs. This study implements Dijkstra's algorithm in Python 3 in order to find the longest path through a graph. This can be done by finding the shortest path through the inverted, non-negative, problem graph. This path will be the path that yields the greatest profit, meaning this path is the one that will give us the most optimal trading strategy.

The greatest profit in the base problem is 252.73€, and in the extended problem, it is 188.27€. From this, it can be concluded that the extensions that have been made, significantly reduce the profit. This is mainly due to the fluctuating limits of inventory, and the limits of buying and selling per month, as they affect the profit over the course of several months, compared to the penalty factor which only affects the last two months.
