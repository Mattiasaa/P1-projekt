Forestil jer, at vi kontrollerer et gaslager—en fysisk enhed, hvori vi kan lagre naturgas—i en given periode på et år. Med “kontrollerer” menes her, at vi ikke selv ejer enheden, men at vi har indgået en kontrakt med ejeren om at leje enheden. Vi er interesserede i at maksimere profitten (eller minimere tabene), som opnås ved at købe og sælge gas i løbet af lejeperioden.
Antag hertil, at vi har en såkaldt forward-kurve for prisen på gas, dvs. en række kon- trakter med modparter på markedet, som giver os forudbestemte priser på gassen til hver tidspunkt i lejeperioden. Til hvert tidspunkt i denne periode kan vi købe eller sælge en be- grænset mængde gas, og der er øvre og nedre grænser på størrelsen af gaslageret (enten fysiske eller kontraktbestemte). Når lejeperioden udløber, køber ejeren af lageret den efter- ladte mængde gas af os. Vi ser desuden bort fra de omkostninger, der måtte være i en sådan lejekontrakt.